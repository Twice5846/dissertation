% Exploring the depths and intricacies of the theory of knowledge led me down a rabbit hole out of which I have yet to climb -- and I despair, but still find solace in laughter.

% As we must conduct every description, make every statement and derive every understanding within the constraints of language, we will forever be held captive by the limits our language(s) impose \parencite{wittgenstein-33}.

% Let us assume knowledge as our ability to differentiate true from false \parencite{gettier-63}.
% Then from the existence of true and false it follows that language -- by my definition -- may express \textit{either} the one \textit{or} the other; not both, not none and nothing else.
% We could then define, under these assumptions, knowledge, as a \enquote{razor of truth}.

% And truth?
% We can not describe the ontological concept of truth because it constitutes itself the fabric that holds our worldly ontology together.
% There is no meta truth.

% We have to believe.
% Trust in God.
% Multiple \enquote{Theories of Truth} exist.
% And all will ultimately ask you to believe.
% One is more ridiculous then the next.

% We could \textit{coherently} build truths a priori from existing truths.
% But what is the root truth? The ultimate truth?
% We will starve on our quest to find it.

% We could employ our ability \textit{correspond} our statements with observable phenomena in the real world.
% But what is the real world and what is our empiricism?
% The clear sky by day is blue.
% That is a true statement for most.
% What about color blind people for whom blue does not exist.

% We could then retreat to \textit{consensus}.
% A statement is true when enough people agree.

% And there, we are one step away from finding our answer.
% Why concern ourselves with the agreement of others.
% There is one truth.
% My truth.
% And it is true because I say it is true.


% The intimate tie between knowledge and language to the point of my confession of considering these equivalent, or at the very least indiscernible, drove me into the frenzied embrace of madness.

\chapter{Preliminary Philosophical Considerations}

None of the words, sentences, statements and insights contained within this dissertation would be of any value without embedding them inside a firm, foundational, explicitly stated epistemic frame.
This first chapter will construct this philosophical bedrock and lay the obligatory groundwork for conducting any further work.

\textit{Note}: The following sections are written in the first person utilizing the inclusive we.
I found it to be most fitting.
You will see why.


\section{Motivation}

The acquisition of knowledge -- the aim of any scientific endeavour -- requires a prior definition of the concept of knowledge.
Only on top of this definition can a subsequent research method be configured and executed.
Only then can a research method claim to contribute to the existing body of knowledge.

I will not explicitly delve into the different epistomelogical positions.
And neither give an overview because it is obsolete.
I will argue my case through which my positioning will become apparent.

I confess, I can -- in the format of this text -- use mere words to communicate my thoughts.
My first request, thus, is for you to conspire with me and accept that what I will attempt to say, can, in fact, be verbalized.
How, otherwise, may the nature of knowledge be examined when the words employed to that end are required to be true and thus contain knowledge themselves?

\textit{Note}: I conciously decided on the term \textit{word}, to differentiate it from \textit{language} and other symbols.
See, also, the \enquote{Triangle of Reference} \parencite{ogden-23}.


\section{The Postulate of Shared Meaning}

To begin, let us face a meta-ontological question and ask how we can define (in words) any concept (including knowledge) that we can grasp with our mind.
Axiomatically, we must accept that our cognition will map the concepts in our mind \textit{appropriately} onto the words we say.
This statement remains a necessary assumption for any well-meaning discussion.
Any attempt to investigate what constitues this appropriateness (i.\,e. circumstances, time, individualities) is futile, because the spoken words are the only gateway to the mind of the speaker \parencite{wittgenstein-53}.
Any attempt to skip the words and move our focus towards our own mind would inevitably lead to methodological solipsism \parencite[136]{putnam-75}, which would render any communication about the concepts contained within the mind redundant.

XXX TODO: Show that evaluating pure truth of linguistic statements is impossible because the definitions of words that are used to express the statements that needs to be evaluated need to be given ad infinitum XXX

% Knowledge for myself is nonsensical.
It follows, then, that words can only obtain \textit{meaning} when they -- not their truth, but rather their sense -- can be validated against something outside the mind of the speaker.
That is, when no one hears me, speaking itself becomes nonsensical \parencite{wittgenstein-33}.
I can call the sky \enquote{blue} today, and \enquote{green} tomorrow and \enquote{red} the day after.
And each statement would be true and would have to pass as knowledge, because there is no way (and no need) to proof it.
Everything I would say under these circumstances becomes true by default.

% Postulate of shared meaning. But how?
We have arrived at my first epistemic show-of-hands: The postulate of \textit{shared meaning}, which states that knowledge can only exist intersubjectively.
Which means, that the words I use to verbalize my mind's concepts must correspond -- when listened to -- to \textit{equivalent} concepts in the mind of the listener.
How can we be sure of that?


\section{The Postulate of Atomicity}

XXX Maybe instead of Atomicity: The Postulate of Pragmatism - knowledge must lead to justified action XXX

% Quick detour over nature and research in natural science
Even with no other human listening to us, we can never speak into the void.
Nature will always be there.
We can not lie to nature nor escape it.
However, nature does not speak any natural language.
Thus, we can not ask it to explain itself.
We may only ever ask: yes or no?
And we do that by executing actions against nature and investigating the resonance.
Over time we can build theories about nature and express them in words under the assumption that other humans interpreting these words have the same understanding and use the same words to describe the phenomena witnessed in nature \parencite[1011b]{aristotle-33}.
This is the basis for any research in the natural sciences.
Because the static and tangible subject of research allows us to build theories and retreat to the methods of fallibilism.
The natural sciences escape any definition of knowledge by this.
Instead, buidling their argument on top of pragmatism.
Can we use the same methodological approach and the same epistemological paradigms when considering a contingent research subject like the human mind \parencite{frank-07}?
See, \enquote{Correspondence Theory of Truth} \parencite{david-22}.

So, again, how can we be sure that the words we speak will evoke the same mental concepts in the listener as in ourselves?
To repel any ad absurdum discussions about the \enquote{nature of reality}, let us also please follow a phenomenological approach to this question \parencite{husserl-02}.
See, \enquote{Twin Earth thought experiment} \parencite{putnam-73}.

One approach is to, again, consider words to be our portal into the human mind (see above) and enter a \textit{dialogue} with other humans about what constitutes knowledge.
But which words -- which knowledge -- will ultimately be accepted?
We will accept that, which is most useful and effective to facilitate successful actions in practical contexts.
This means, that truth is not an abstract, static concept but is determined by the practical outcomes of holding certain beliefs.
See, the \enquote{Pragmatic Theory of Truth} \parencite{capps-23}.

This leads to my second epistemic unveiling: The postulate of \textit{atomicity}, which states that knowledge exists sui generis.
Knowledge is not contingent upon external validation or empirical evidence, but on its inherent capacity to produce practical outcomes.
We may not break knowledge down into further components like justification and belief.
See, \enquote{Knowledge and its Limits} \parencite{williamson-02}.

But what is a \enquote{problem} and a \enquote{successful action}?
Epistemology stops here and answers with \textit{belief}.
We would have to enter the realm of existentialism to finds answers, which I will do -- briefly.
The question is, how can we construct a bridge to close the gap between the pragmatism required for knowledge and the evaluation of proof of pragmatic ideas when we have no mirror -- like nature -- against which to check our knowledge.


\section{The Postulate of the Will}

When knowledge can only be expressed in words, which it has to, as we have seen above.
Only under the assumption that a limited number of word configurations exists, can we speak of \enquote{contributions to the body of knowledge}.
It appears, though, that this assumption does not hold.
First, even the countable set of words, that we call our vocabulary, grows and I can see no inherent growth throttling of this set - a priori or empirically.
Second, the set of combinations of words -- their configuration -- is infinite.
Quod erat demonstrandum.

We could only bypass this by accepting that our mind itself is limited in its capacity to create pictures out of these word configurations and that eventually clusters of pictures emerge that are in themselves \enquote{similar enough} to speak of equivalency between them, inside the clusters.
This does however require a similarity metric - not necessarily quantifiably stated.
How? With the aim of advancing one's will (Kant, Schopenhauer, Nietzsche, Sartre). In this sense, as the will is continuously evolving and at any point there exists one thinkable yet unattainable optimal solution to advancing it, knowledge is in any point limited but required to be extended when considered on a continuum.

This leads me to my final epistemic unraveling and my confession to following a perspectivistic approach to anything that comes after this:
The postulate of \textit{the will}, which states that it is our individual will to interpret and give meaning to our experiences and actions \parencite{schopenhauer-09}.

To conclude, I can only try to communicate my own \textit{perspective} and try to argue, to the best of my ability, how pragmatically useful my thoughts are -- for me.
I can not -- nor will I try -- to convince the reader of anything that is outside the scope of my perspective.

XXX TODO: Try to differentiate or integrate \textit{usefulness} in i.\,e. information systems XXX


\section{Summary}

I wanted to define what knowledge is.
I have first followed a philosophy of language approach and asked how we can define, using words, \textit{anything}.
We have seen that we need shared meaning is a requirement.

How can we proof that we actually mean the same thing by the words we say?
I have then borrowed methods from the natural sciences and stated that it is the pragmatic content that makes knowledge knowledge.
This means, knowledge is not something that statements possess but rather that knowledge is the thing that has the most benefit for our actions.
This also means that we can never know anything for certain.
We can just know \textit{enough} for a specific \textit{goal}.

To advance these thoughts one step further and answer what pragmatism is and by what metric it can be evaluated, I have refuted blind faith and looked into the philosophy of existentialism and employed the concept of the \textit{will} as propagated by prominent philosophers.
Will is something highly individualistic that can not nor should it be further investigated.

In conclusion, I must follow a perspectivistic approach in this dissertation and tell how things appear to \textit{me} and what insights \textit{I} need to advance \textit{my} will.
Fear not, though.
As we are all humans and my readers will supposedly be scholars like myself, it is fair to assume that my insights will stem from an appropriately similar will and thus exhibit enough congruency for them to be useful enough for my readers to consume them.

XXX TODO: Show that the quest for originality or generalization is unnecessary: Even if the meaning of what I say has been said before, I have now used different words. That is original (enough). What scope of generalization are we talking about? XXX

% Kann nur meine eigene Perspektive vermitteln.
% Andere können dann damit ihre eigene erweitern.
% Abzugrenzen von den Naturwissenschaften, denn der Natur sind wir alle unterworfen.

% That is, to consider even knowledge to be a perception.

% without it these words themselves would have to be questioned and thinking itself would become impossible.

% I want to foreground \textit{perceive} as to remove any notion of the unncessarily constructed idea of a \enquote{real world}.
% This way, we can avoid the detour of the realists, who themselves ultimately reduce reality to a subset of perceptions -- how could they not -- and focus on ourselves.

% We can only know what can be said.
% Nothing exists beyond declarative knowledge.
% This is a contradiction.

% % Knowledge can not be defined from the perspective of knowledge itself.
% To specify what knowledge is and what it is not, we would have to enter a discussion about meta-knowledge.
% This in itself requires knowledge about meta-knowledge which leads to a circulus in demonstrando.

% % Knowledge as a form of belief; and believes are final.
% We could consider knowledge to be a subset of \textit{belief};
% Because a) we believe everything we know and b) we are not obligated nor can we further specifiy believes.
% Belief is a final concept. \enquote{I believe} can not be falsified.

% % Justified true believe
% How do we distinguish our believes into faith and knowledge?
% A proposed solution is the long-held concept of justified true believe:
% We know something when we believe it to be true and can justify why we believe it.
% This raises the question of what constitutes justification \parencite{gettier-63}.
% To give a justification we would have to also justify why it justifies our believe which requires us to know everything beforehand to rule out contradictions to our justification.

% ----------

% As we have seen in the previous chapter, considering what knowledge is on an individual level is irrelevant as it can never be proven or falsified.
% If I were to say, for example: \enquote{I know that the sky is green}.
% You could not agree or disagree without knowing what I \textit{mean} by each of the words including what \textit{knowing} means for me.
% This is obvious but necessary for any further investigation.
% Because this simple claim has several profound implications:
% \begin{enumerate}
%   \item There exists a common world accessible by more than one.
%   \item One can only say what can be understood.
%   \item One can only know what others can know.
%   \item Not everything can be said.
% \end{enumerate}
% As we can see our ability to acquire knowledge is heavily constrained by the understanding and interpretation of others.

% And let any statement be \textit{either} true \textit{or} false by excluding all nonsensical and senseless statements \parencite{wittgenstein-33}.
% And let the \enquote{world} be expressable by (linguistic) statements.
% Then the quest for knowledge leads the seeker down the path of evaluating the truth of statements.

% Here, the essential problem and the motivation for this chapter reveals itself.
% Asking of any statement to uncover its truth asks in return of the questioner to disclose \textit{what} will satisfy their request for truth.
% \textit{Example}: \enquote{The sky is blue} -- is this statement true?
% Ignore the ontological implications that things exist, things can possess or be other things, that properties exist, that there are different types of relationships between things, etc.
% The truth of this simple statement can not be evaluated.
% The questioner would have to elaborate on what they mean by sky (i.e. by day or by night, on earth or the moon, the physical sky or the metaphorical one).
% They would have to give a definition for blue.
% What does it \textit{mean} for something to \enquote{be blue}?

% However, the nature of truth -- and thus the satisfaction criteria for truth -- can not universally be described.
% There is no meta truth.

% \begin{enumerate}
%   \item \textit{Postulate of the Ontology of Knowledge}: Knowledge exists sui generis.
%   \item \textit{Postulate of Shared Meaning}: Knowledge exists only intersubjectively.
% \end{enumerate}